\documentclass[12pt,a4paper]{article}
\usepackage[utf8]{inputenc}
\usepackage[T1]{fontenc}
\usepackage[polish]{babel}
\usepackage{geometry}
\usepackage{graphicx}
\usepackage{listings}
\usepackage{xcolor}
\usepackage{hyperref}
\usepackage{booktabs}
\usepackage{array}
\usepackage{tikz}
\usetikzlibrary{arrows.meta, positioning, shapes.geometric}

\geometry{margin=2.5cm}

% Kolory dla listingów
\definecolor{codegreen}{rgb}{0,0.6,0}
\definecolor{codegray}{rgb}{0.5,0.5,0.5}
\definecolor{codepurple}{rgb}{0.58,0,0.82}
\definecolor{backcolour}{rgb}{0.95,0.95,0.92}

\lstdefinestyle{mystyle}{
    backgroundcolor=\color{backcolour},   
    commentstyle=\color{codegreen},
    keywordstyle=\color{magenta},
    numberstyle=\tiny\color{codegray},
    stringstyle=\color{codepurple},
    basicstyle=\ttfamily\footnotesize,
    breakatwhitespace=false,         
    breaklines=true,                 
    captionpos=b,                    
    keepspaces=true,                 
    numbers=left,                    
    numbersep=5pt,                  
    showspaces=false,                
    showstringspaces=false,
    showtabs=false,                  
    tabsize=2,
    frame=single
}
\lstset{style=mystyle}

\title{
    \vspace{-2cm}
    \textbf{Gra w Statki} \\
    \large Sieciowa gra dla dwóch graczy \\
    \vspace{0.5cm}
    \normalsize Programowanie Współbieżne 2025 -- Zadanie 11
}
\author{Kajetan Lach}
\date{\today}

\begin{document}

\maketitle

\tableofcontents
\newpage

% ============================================================================
\section{Sformułowanie zadania}
% ============================================================================

\subsection{Opis ogólny}

Celem zadania jest zaimplementowanie sieciowej gry w statki (Battleship) dla dwóch graczy z zachowaniem zasad programowania współbieżnego. Rozwiązanie umożliwia rozgrywkę użytkownikom działającym na osobnych komputerach lub w osobnych procesach na jednej maszynie z wykorzystaniem gniazd sieciowych (sockets).

\subsection{Zasady gry}

\subsubsection{Plansza}
Gra odbywa się na dwóch kwadratowych planszach o rozmiarze $10 \times 10$ pól. Każdy gracz widzi:
\begin{itemize}
    \item \textbf{Własną planszę} -- z rozmieszczonymi statkami
    \item \textbf{Planszę strzałów} -- widok pola przeciwnika z oznaczonymi trafieniami i pudłami
\end{itemize}

\subsubsection{Flota}
Każdy gracz dysponuje następującą flotą:
\begin{center}
\begin{tabular}{|c|c|c|}
\hline
\textbf{Typ statku} & \textbf{Rozmiar (pola)} & \textbf{Ilość} \\
\hline
Jednomasztowiec & 1 & 4 \\
Dwumasztowiec & 2 & 3 \\
Trzymasztowiec & 3 & 2 \\
Czteromasztowiec & 4 & 1 \\
\hline
\textbf{Razem} & \textbf{20 pól} & \textbf{10 statków} \\
\hline
\end{tabular}
\end{center}

\subsubsection{Rozmieszczenie statków}
\begin{itemize}
    \item Statki mogą być ustawione w pionie lub poziomie
    \item Statki \textbf{nie mogą się stykać} bokami ani rogami
    \item Pozycje statków są generowane losowo na początku gry
\end{itemize}

\subsubsection{Przebieg rozgrywki}
\begin{enumerate}
    \item \textbf{Start:} Grę rozpoczyna losowo wybrany gracz
    \item \textbf{Tura:} Gracze wykonują ruchy naprzemiennie
    \item \textbf{Strzał:} Gracz wybiera pole na planszy strzałów
    \item \textbf{Wynik strzału:}
    \begin{itemize}
        \item \textit{Trafienie} -- gracz kontynuuje turę
        \item \textit{Pudło} -- tura przechodzi do przeciwnika
        \item \textit{Zatopienie} -- wszystkie pola statku zostały trafione
    \end{itemize}
    \item \textbf{Wygrana:} Wygrywa gracz, który pierwszy zatopi wszystkie statki przeciwnika
\end{enumerate}

% ============================================================================
\section{Architektura systemu}
% ============================================================================

\subsection{Model klient-serwer}

Aplikacja wykorzystuje architekturę klient-serwer:

\begin{center}
\begin{tikzpicture}[
    node distance=3cm,
    server/.style={rectangle, draw, fill=blue!20, minimum width=3cm, minimum height=1.5cm, align=center},
    client/.style={rectangle, draw, fill=green!20, minimum width=2.5cm, minimum height=1.2cm, align=center},
    arrow/.style={-Stealth, thick}
]
    \node[server] (server) {Serwer\\(server.py)};
    \node[client, left=of server] (client1) {Klient 1\\(client.py)};
    \node[client, right=of server] (client2) {Klient 2\\(client.py)};
    
    \draw[arrow, <->] (client1) -- node[above] {TCP/IP} (server);
    \draw[arrow, <->] (server) -- node[above] {TCP/IP} (client2);
\end{tikzpicture}
\end{center}

\subsection{Komponenty systemu}

\begin{description}
    \item[server.py] Serwer gry -- zarządza połączeniami, koordynuje rozgrywkę, przechowuje stan gry
    \item[client.py] Klient z GUI -- interfejs graficzny Tkinter, obsługa interakcji użytkownika
    \item[protocol.py] Protokół komunikacji -- definicje wiadomości, kodowanie/dekodowanie JSON
    \item[game\_logic.py] Logika gry -- plansze, statki, zasady, sprawdzanie warunków wygranej
\end{description}

\subsection{Współbieżność}

System wykorzystuje wielowątkowość do obsługi równoczesnych operacji:

\begin{itemize}
    \item \textbf{Serwer:} Każdy klient obsługiwany w osobnym wątku
    \item \textbf{Klient:} Wątek główny (GUI) + wątek odbierania wiadomości
    \item \textbf{Synchronizacja:} Blokada (lock) chroni dostęp do współdzielonego stanu gry
\end{itemize}

% ============================================================================
\section{Schemat komunikacji}
% ============================================================================

\subsection{Protokół transportowy}

Komunikacja odbywa się przez protokół TCP/IP z wykorzystaniem gniazd (sockets):
\begin{itemize}
    \item \textbf{Port domyślny:} 5000
    \item \textbf{Format danych:} JSON z nagłówkiem długości
    \item \textbf{Kodowanie:} UTF-8
\end{itemize}

\subsection{Format wiadomości}

Każda wiadomość składa się z:
\begin{enumerate}
    \item \textbf{Nagłówka} (8 bajtów) -- długość danych JSON
    \item \textbf{Danych} -- obiekt JSON z polami \texttt{type} i \texttt{data}
\end{enumerate}

\begin{lstlisting}[language=Python, caption=Struktura wiadomości]
{
    "type": "shoot",       # Typ wiadomosci
    "data": {              # Dane zalezne od typu
        "row": 5,
        "col": 3
    }
}
\end{lstlisting}

\subsection{Typy wiadomości}

\begin{center}
\begin{tabular}{|l|c|p{7cm}|}
\hline
\textbf{Typ} & \textbf{Kierunek} & \textbf{Opis} \\
\hline
\texttt{CONNECTED} & S $\rightarrow$ K & Potwierdzenie połączenia z ID gracza i planszą \\
\texttt{GAME\_START} & S $\rightarrow$ K & Rozpoczęcie gry, informacja o turze \\
\texttt{SHOOT} & K $\rightarrow$ S & Strzał gracza (row, col) \\
\texttt{SHOT\_RESULT} & S $\rightarrow$ K & Wynik własnego strzału \\
\texttt{OPPONENT\_SHOT} & S $\rightarrow$ K & Informacja o strzale przeciwnika \\
\texttt{GAME\_OVER} & S $\rightarrow$ K & Koniec gry z wynikiem \\
\texttt{PLAY\_AGAIN} & K $\rightarrow$ S & Żądanie rematchu \\
\texttt{DISCONNECT} & K $\leftrightarrow$ S & Rozłączenie \\
\texttt{OPPONENT\_DISCONNECTED} & S $\rightarrow$ K & Przeciwnik opuścił grę \\
\hline
\end{tabular}
\end{center}

\textit{Legenda: S -- Serwer, K -- Klient}

\subsection{Diagram sekwencji -- rozgrywka}

\begin{center}
\begin{tikzpicture}[
    scale=0.9,
    transform shape,
    participant/.style={rectangle, draw, fill=gray!20, minimum width=2cm, minimum height=0.7cm},
    message/.style={-Stealth, thick},
    note/.style={rectangle, draw, dashed, fill=yellow!20, align=left, font=\footnotesize}
]
    % Uczestnicy
    \node[participant] (c1) at (0,0) {Klient 1};
    \node[participant] (s) at (5,0) {Serwer};
    \node[participant] (c2) at (10,0) {Klient 2};
    
    % Linie życia
    \draw[dashed] (0,-0.5) -- (0,-12);
    \draw[dashed] (5,-0.5) -- (5,-12);
    \draw[dashed] (10,-0.5) -- (10,-12);
    
    % Wiadomości
    \draw[message] (0,-1) -- node[above, font=\scriptsize] {CONNECT} (5,-1);
    \draw[message] (5,-1.5) -- node[above, font=\scriptsize] {CONNECTED} (0,-1.5);
    
    \draw[message] (10,-2) -- node[above, font=\scriptsize] {CONNECT} (5,-2);
    \draw[message] (5,-2.5) -- node[above, font=\scriptsize] {CONNECTED} (10,-2.5);
    
    \draw[message] (5,-3.5) -- node[above, font=\scriptsize] {GAME\_START} (0,-3.5);
    \draw[message] (5,-4) -- node[above, font=\scriptsize] {GAME\_START} (10,-4);
    
    \node[note] at (7,-5) {Gracz 1 zaczyna};
    
    \draw[message] (0,-6) -- node[above, font=\scriptsize] {SHOOT(3,5)} (5,-6);
    \draw[message] (5,-6.5) -- node[above, font=\scriptsize] {SHOT\_RESULT(hit)} (0,-6.5);
    \draw[message] (5,-7) -- node[above, font=\scriptsize] {OPPONENT\_SHOT} (10,-7);
    
    \node[note] at (7,-8) {Trafienie -- Gracz 1 kontynuuje};
    
    \draw[message] (0,-9) -- node[above, font=\scriptsize] {SHOOT(4,5)} (5,-9);
    \draw[message] (5,-9.5) -- node[above, font=\scriptsize] {SHOT\_RESULT(miss)} (0,-9.5);
    \draw[message] (5,-10) -- node[above, font=\scriptsize] {OPPONENT\_SHOT} (10,-10);
    
    \node[note] at (7,-11) {Pudło -- tura przechodzi};
    
\end{tikzpicture}
\end{center}

% ============================================================================
\section{Instrukcja użytkowania}
% ============================================================================

\subsection{Wymagania systemowe}

\begin{itemize}
    \item Python 3.8 lub nowszy
    \item Biblioteka Tkinter (standardowo wbudowana w Python)
    \item Dostęp do sieci (dla gry między komputerami)
\end{itemize}

\subsection{Uruchomienie serwera}

\begin{lstlisting}[language=bash, caption=Uruchomienie serwera]
cd zad11
python server.py
\end{lstlisting}

Opcjonalne parametry:
\begin{itemize}
    \item \texttt{-H / --host} -- adres nasłuchiwania (domyślnie: 0.0.0.0)
    \item \texttt{-p / --port} -- port (domyślnie: 5000)
\end{itemize}

Przykład z niestandardowym portem:
\begin{lstlisting}[language=bash]
python server.py -H 192.168.1.100 -p 8080
\end{lstlisting}

\subsection{Uruchomienie klienta}

\begin{lstlisting}[language=bash, caption=Uruchomienie klienta]
python client.py
\end{lstlisting}

W oknie aplikacji:
\begin{enumerate}
    \item Wpisz adres serwera (np. \texttt{localhost} lub IP serwera)
    \item Wpisz port (np. \texttt{5000})
    \item Kliknij przycisk \textbf{Połącz}
    \item Poczekaj na drugiego gracza
    \item Gdy gra się rozpocznie, klikaj na pola planszy strzałów
\end{enumerate}

\subsection{Interfejs graficzny}

\begin{center}
\begin{tabular}{|c|l|}
\hline
\textbf{Element} & \textbf{Opis} \\
\hline
Lewa plansza & Twoja flota (widoczne statki) \\
Prawa plansza & Plansza strzałów (pole przeciwnika) \\
Wskaźnik tury & Informuje czyja jest tura \\
Pasek statusu & Komunikaty o strzałach i zdarzeniach \\
\hline
\end{tabular}
\end{center}

\subsection{Legenda kolorów}

\begin{center}
\begin{tabular}{|c|l|}
\hline
\textbf{Kolor} & \textbf{Znaczenie} \\
\hline
Niebieski & Woda (nieodkryte pole) \\
Stalowy & Statek (widoczny tylko na własnej planszy) \\
Czerwony & Trafienie \\
Szary & Pudło \\
Fioletowy & Zatopiony statek \\
\hline
\end{tabular}
\end{center}

% ============================================================================
\section{Obsługa sytuacji błędnych}
% ============================================================================

\subsection{Błędy połączenia}

\begin{tabular}{|p{5cm}|p{8cm}|}
\hline
\textbf{Sytuacja} & \textbf{Reakcja systemu} \\
\hline
Serwer niedostępny & Komunikat: \textit{„Nie można połączyć z serwerem"} \\
\hline
Timeout połączenia & Komunikat po 5 sekundach oczekiwania \\
\hline
Gra pełna (2 graczy) & Komunikat: \textit{„Gra jest pełna"} \\
\hline
\end{tabular}

\subsection{Błędy w trakcie gry}

\begin{tabular}{|p{5cm}|p{8cm}|}
\hline
\textbf{Sytuacja} & \textbf{Reakcja systemu} \\
\hline
Strzał w zajęte pole & Komunikat: \textit{„To pole było już ostrzeliwane"} \\
\hline
Strzał nie w swojej turze & Blokada interakcji z planszą strzałów \\
\hline
Rozłączenie przeciwnika & Komunikat + oczekiwanie na nowego gracza \\
\hline
Utrata połączenia z serwerem & Automatyczne rozłączenie + komunikat \\
\hline
\end{tabular}

\subsection{Zamknięcie aplikacji}

\begin{itemize}
    \item Zamknięcie okna klienta powoduje bezpieczne rozłączenie z serwerem
    \item Serwer informuje pozostałego gracza o rozłączeniu przeciwnika
    \item Serwer można zatrzymać przez Ctrl+C (sygnał SIGINT)
\end{itemize}

% ============================================================================
\section{Testowanie}
% ============================================================================

Projekt zawiera automatyczne testy weryfikujące poprawność implementacji:

\begin{lstlisting}[language=bash, caption=Uruchomienie testów]
python test_game.py
\end{lstlisting}

Testy sprawdzają:
\begin{itemize}
    \item Generowanie planszy i rozmieszczanie statków
    \item Protokół komunikacji (serializacja/deserializacja)
    \item Połączenie z serwerem i wymianę wiadomości
    \item Prawidłowość zmiany tur
\end{itemize}

\end{document}
